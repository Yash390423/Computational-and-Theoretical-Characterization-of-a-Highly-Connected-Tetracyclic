\documentclass[12pt]{article}
\usepackage[margin=1in]{geometry} % Sets 1-inch margins
\usepackage{amsmath}              % For math environments
\usepackage{amssymb}              % For math symbols
\usepackage{graphicx}             % For images
\usepackage{braket}               % For the |L - lambda I| set
\usepackage{adjustbox}            % To scale the large matrix

% --- PACKAGE FOR HEADERS ---
\usepackage{fancyhdr}
% --- PACKAGE FOR LINE SPACING ---
\usepackage{setspace} % <-- ADDED THIS PACKAGE

% Header settings
\fancyhf{} % clear all header and footer fields
\fancyhead[L]{ChE 209: Final Project} % Left Header
\fancyhead[R]{Derivation for g-factor}  % Right Header
\fancyfoot[C]{\thepage}               % Page number in center footer
\renewcommand{\headrulewidth}{0.4pt} % adds a line under the header

\begin{document}
\thispagestyle{empty} % --- This makes the first page (title/index) clean ---

% --- TITLE BLOCK ---
\begin{center}
    \vspace*{0.5cm} % Space from top
    
    {\LARGE  \textit{Course: ChE 209: Introduction to Soft Matters \& Polymers}}
    
    \vspace{1cm}
    
    {\large \textit{Final Project: Part II}}
    
    \vspace{1.5cm}
    
    {\LARGE Derivation for the theoretical contraction factors $g(G_{\infty}, G_{\text{tree}\infty} )$}
    
    \vspace{1.5cm}
    
    {\large Submitted by:}
    
    \vspace{0.5cm}
    
\begin{center}
    \begin{minipage}{0.8\textwidth} 
        \large 
        \centering 
        \setstretch{1.3} % <-- Added to fix spacing
        Vaatsalya Sahu $\vert$ 240008034 \\
        Vantakula Sujan Kumar $\vert$ 240008035 \\
        Vinay Gound $\vert$ 240008036 \\
        Vishal Shakya $\vert$ 240008037 \\
        Yash Pankaj Chaudhari $\vert$ 240008038 \\
        Yogen Dand $\vert$ 240008039
    \end{minipage}
\end{center}
% --- END OF TITLE BLOCK ---

\end{center} % <-- THIS IS THE CORRECTED LINE (moved from below)

\tableofcontents 

% --- Page 2 starts here ---
\newpage 
\pagestyle{fancy} % --- The header style now applies from this page forward ---

\section{General Formula}

The asymptotic g-factor $g(G_{\infty})$ is calculated using the following formula:
$$
g(G_{\infty}) = \frac{3}{e(G)^2} \left( \text{Tr}(L^+(G)) + \frac{1}{3}\text{loops}(G) - \frac{1}{6} \right)
$$
Where:
\begin{itemize}
    \item $e(G)$ is the number of edges in graph $G$.
    \item $v(G)$ is the number of vertices in graph $G$.
    \item $\text{loops}(G)$ is the cycle rank of graph $G$, given by:
    $$
    \text{loops}(G) = e(G) - v(G) + 1
    $$
    \item $\text{Tr}(L^+(G))$ is the trace of the Moore-Penrose pseudoinverse of the normalized graph Laplacian, $L(G)$. This is found by summing the reciprocals of all non-zero eigenvalues ($\lambda_i$) of the $L(G)$ matrix:
    $$
    \text{Tr}(L^+(G)) = \sum_{\lambda_i \neq 0} \frac{1}{\lambda_i}
    $$
    \item The normalized graph Laplacian $L(G)$ is a $v(G) \times v(G)$ matrix defined as:
    $$
    L_{ij} = \begin{cases} 
        1 - \frac{2 \times (\text{no. of loop edges})}{deg(i)} & \text{if } i = j \\
        \frac{-R}{\sqrt{deg(i) deg(j)}} & \text{if } v_i, v_j \text{ are joined by } R \text{ edges} \\
        0 & \text{otherwise}
    \end{cases}
    $$
\end{itemize}

\section{Calculation for Alpha-like Polymer ($G_{\alpha}$)}

\begin{center}
    % --- IMAGE ADDED HERE ---
    % Make sure you upload a file named "alpha_graph.png" to your project
    \includegraphics[width=0.6\textwidth]{alpha_graph.png}
\end{center}

For this graph, we have:
\begin{itemize}
    \item Vertices: $v(G_{\alpha}) = 6$
    \item Edges: $e(G_{\alpha}) = 9$
\end{itemize}
All vertices have a degree of 3 ($deg(v_i) = 3$). A $6 \times 6$ matrix will be formed.

\subsection{Laplacian Matrix $L(G_{\alpha})$}
Since this polymer has no loop edges, all diagonal entries are $L_{ii} = 1 - 0 = 1$.
$$
L_{11} = L_{22} = L_{33} = L_{44} = L_{55} = L_{66} = 1
$$
For the non-diagonal entries for connected vertices (with $R=1$), the value is:
$$
L_{ij} = \frac{-1}{\sqrt{deg(i) deg(j)}} = \frac{-1}{\sqrt{3 \times 3}} = -\frac{1}{3}
$$
Calculating the entries for the first row (vertex $v_1$) as an example:
\begin{align*}
L_{12} &= -1/3 \quad (\text{connected to } v_2) \\
L_{13} &= 0 \quad (\text{not connected to } v_3) \\
L_{14} &= -1/3 \quad (\text{connected to } v_4) \\
L_{15} &= 0 \quad (\text{not connected to } v_5) \\
L_{16} &= -1/3 \quad (\text{connected to } v_6)
\end{align*}
Similarly, we can calculate all other entries. The full $6 \times 6$ matrix is:
$$
\mathbf{L}(G_{\alpha}) = 
\begin{pmatrix}
1 & -1/3 & 0 & -1/3 & 0 & -1/3 \\
-1/3 & 1 & -1/3 & 0 & -1/3 & 0 \\
0 & -1/3 & 1 & -1/3 & 0 & -1/3 \\
-1/3 & 0 & -1/3 & 1 & -1/3 & 0 \\
0 & -1/3 & 0 & -1/3 & 1 & -1/3 \\
-1/3 & 0 & -1/3 & 0 & -1/3 & 1
\end{pmatrix}
$$

\subsection{Eigenvalues and Trace}
To find the eigenvalues, we solve $|L(G_{\alpha}) - \lambda I| = 0$, where $I$ is the identity matrix. 
After computation, the eigenvalues ($\lambda$) are:
$$
\lambda = \{0, 1, 1, 1, 1, 2\}
$$
Now, to find $\text{Tr}(L^+(G_{\alpha}))$, we sum the reciprocals of the non-zero eigenvalues:
$$
\text{Tr}(L^+(G_{\alpha})) = \frac{1}{1} + \frac{1}{1} + \frac{1}{1} + \frac{1}{1} + \frac{1}{2} = 1 + 1 + 1 + 1 + 0.5 = 4.5
$$

\subsection{g-factor $g(G_{\alpha})$}
First, we find the cycle rank:
$$
\text{loops}(G_{\alpha}) = e(G) - v(G) + 1 = 9 - 6 + 1 = 4
$$
Now, we calculate the g-factor:
\begin{align*}
g(G_{\alpha}) &= \frac{3}{e(G_{\alpha})^2} \left( \text{Tr}(L^+(G_{\alpha})) + \frac{1}{3}\text{loops}(G_{\alpha}) - \frac{1}{6} \right) \\
&= \frac{3}{9^2} \left( 4.5 + \frac{1}{3}(4) - \frac{1}{6} \right) \\
&= \frac{3}{81} \left( \frac{9}{2} + \frac{7}{6} \right) \\
&= \frac{3}{81} \left( \frac{27}{6} + \frac{7}{6} \right) = \frac{3}{81} \left( \frac{34}{6} \right) \\
&= \frac{3}{81} \left( \frac{17}{3} \right) \\
&= \frac{17}{81}
\end{align*}

\section{Calculation for Tree Graph ($G_{\text{tree}}$)}

\begin{center}
    % --- IMAGE ADDED HERE ---
    % Make sure you upload a file named "tree_graph.png" to your project
    \includegraphics[width=0.5\textwidth]{tree_graph.png}
\end{center}

For this graph, we have:
\begin{itemize}
    \item Vertices: $v(G_{\text{tree}}) = 10$
    \item Edges: $e(G_{\text{tree}}) = 9$
\end{itemize}
A $10 \times 10$ matrix will be made.

\subsection{Laplacian Matrix $L(G_{\text{tree}})$}
$$
\mathbf{L}(G_{\text{tree}}) = 
\begin{adjustbox}{width=1\textwidth}
$
\begin{pmatrix}
1 & -1/3 & -1/3 & -1/3 & 0 & 0 & 0 & 0 & 0 & 0 \\
-1/3 & 1 & 0 & 0 & -1/\sqrt{3} & -1/\sqrt{3} & 0 & 0 & 0 & 0 \\
-1/3 & 0 & 1 & 0 & 0 & 0 & -1/\sqrt{3} & -1/\sqrt{3} & 0 & 0 \\
-1/3 & 0 & 0 & 1 & 0 & 0 & 0 & 0 & -1/\sqrt{3} & -1/\sqrt{3} \\
0 & -1/\sqrt{3} & 0 & 0 & 1 & 0 & 0 & 0 & 0 & 0 \\
0 & -1/\sqrt{3} & 0 & 0 & 0 & 1 & 0 & 0 & 0 & 0 \\
0 & 0 & -1/\sqrt{3} & 0 & 0 & 0 & 1 & 0 & 0 & 0 \\
0 & 0 & -1/\sqrt{3} & 0 & 0 & 0 & 0 & 1 & 0 & 0 \\
0 & 0 & 0 & -1/\sqrt{3} & 0 & 0 & 0 & 0 & 1 & 0 \\
0 & 0 & 0 & -1/\sqrt{3} & 0 & 0 & 0 & 0 & 0 & 1
\end{pmatrix}
$
\end{adjustbox}
$$

\subsection{Eigenvalues and Trace}
After computation, the 10 eigenvalues are:
\begin{align*}
\lambda = \Big\{ & 0, 1, 1, 1, 1, 2, \\
                 & 1-\sqrt{\frac{2}{3}}, 1+\sqrt{\frac{2}{3}}, 1-\sqrt{\frac{2}{3}}, 1+\sqrt{\frac{2}{3}} \Big\}
\end{align*}
We sum the reciprocals of all non-zero eigenvalues to get $\text{Tr}(L^+(G_{\text{tree}}))$.
\begin{align*}
\text{Sum of } \frac{1}{1} \text{ (4 times)} &= 4 \\
\text{Sum of } \frac{1}{2} &= 0.5 \\
\text{Sum of reciprocal pairs:} \quad 2 \times \left( \frac{1}{1 - \sqrt{2/3}} + \frac{1}{1 + \sqrt{2/3}} \right) &= 2 \times \left( \frac{(1 + \sqrt{2/3}) + (1 - \sqrt{2/3})}{(1 - 2/3)} \right) \\
&= 2 \times \left( \frac{2}{1/3} \right) = 2 \times 6 = 12
\end{align*}
Total trace (matching your note):
$$
\text{Tr}(L^+(G_{\text{tree}})) = 4 + 0.5 + 12 = 16.5
$$

\subsection{g-factor $g(G_{\text{tree}})$}
The cycle rank is:
$$
\text{loops}(G_{\text{tree}}) = e(G) - v(G) + 1 = 9 - 10 + 1 = 0 % <-- TYPO FIXED HERE
$$
Now, we calculate the g-factor:
\begin{align*}
g(G_{\text{tree}}) &= \frac{3}{e(G_{\text{tree}})^2} \left( \text{Tr}(L^+(G_{\text{tree}})) + \frac{1}{3}\text{loops}(G_{\text{tree}}) - \frac{1}{6} \right) \\
&= \frac{3}{9^2} \left( 16.5 + \frac{1}{3}(0) - \frac{1}{6} \right) \\
&= \frac{3}{81} \left( 16.5 - \frac{1}{6} \right) \\
&= \frac{3}{81} \left( \frac{33}{2} - \frac{1}{6} \right) \\
&= \frac{3}{81} \left( \frac{99}{6} - \frac{1}{6} \right) \\
&= \frac{3}{81} \left( \frac{98}{6} \right) = \frac{3}{81} \left( \frac{49}{3} \right) \\
&= \frac{49}{81}
\end{align*}

\section{Final Relative g-factor}
The relative g-factor is the ratio of $g(G_{\alpha})$ to $g(G_{\text{tree}})$:
$$
g_{\text{relative}} = \frac{g(G_{\alpha})}{g(G_{\text{tree}})} = \frac{17/81}{49/81} = \boxed{\frac{17}{49}}
$$

\end{document}
